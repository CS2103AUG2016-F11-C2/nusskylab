\chapter{Submission} \label{submission}

Students in Orbital are supposed to report their progress regarding their project for each milestone via \textit{Submission}. Basically a submission contains 3 parts:

\begin{itemize}
  \item Read me: highlights what are the changes and new features in the project.
  \item Project Log: a summary of work done during the phase and time used for each task.
  \item Video link: a link to a video introducing the project to evaluators.
\end{itemize}

As the structure of \textit{Read me} and \textit{Project Log} is free and we should allow creativity of students when it comes to describing their own projects, we decided to support rich text for submissions and there are some issues coming along with this decision such as SQL injection and XSS attacks. What is more, during use of Skylab, many good suggestions regarding user experience were brought up by students and advisers user, like uploading of images and auto-expanding textareas during editing.

\section{Handling of rich text}
We used TinyMCE to support rich text editing feature in Skylab as it is a very popular WYSIWYG editor with a rich set of features\cite{citation10}. There are some libraries with markdown syntax supported such as EpicEditor, Vue.js and Hallo.js which are more lightweight. However, as Skylab is built for freshmen to get more hands-on experience with coding, we do not expect students to be equipped with much prior knowledge such as markdown syntax. In the contrast, markup based editors such as TinyMCE and CKEditor do not require any learning and are easily to get started with. What is more, the large community using TinyMCE has made various plug-ins available for different features. There are also quite resources online about integrating TinyMCE editor in a Rails project. Therefore, we decided to use TinyMCE for \textit{Submissions}' rich text support.

One particular disadvantage of using TinyMCE is that the size of this editor is pretty large and loading of the page will be slowed down because of it. Therefore, TinyMCE related resources is only loaded if the page is requiring rich text editing. This is done by configuration to disable auto loading of all JavaScript, which is the default behavior of Rails. In this way, most pages can stilled be loaded in a very short time.

Rails has built-in checking against SQL injection attacks and therefore Skylab is safe from such attacks when storing submissions' contents\cite{citation11}. However, there is still currently a known bug in the implementation when it comes to viewing of submissions. As contents like \textit{Read me} and \textit{Project Log} should be rendered as rich text, it is possible for students to perform XSS attacks by injecting executable JavaScript code in the submission. This sort of vulnerabilities will be fixed in the future.

\section{Usability}

Skylab is a software engineering project and therefore improvement in user experience is one key part when it comes to implementation. During the use of Skylab, many suggestions were brought up by students and advisers about usability. And by addressing these issues, Skylab is serving users better with a smoother user experience.

\subsection{Target Milestone Selection}

When Skylab was used for the first time, students were expected to choose the target milestone, which the submission is for. However, many students reported that it is just a redundant step as every time they will only submit to the currently active milestone. After hearing this, a quick fix of automatically selecting the current milestone for students using JavaScript functions was done, while the manual selection is still possible. And then during the \textit{Adviser Focus Group Meeting}, some advisers further pointed out that the selection should not even be presented to users as it is of completely no use and therefore the whole selection was completely removed from Skylab after the meeting by moving the task to the backend logic. 

\subsection{Rich Text Editing}

As quite some students want to insert image to \textit{Read me} sections, an image uploading feature was soon added to submission page for users' convenience. Behind the scene Skylab is using a third-party API from Imgur. This was done mainly for 2 reasons: using Imgur is relatively easy to implement; we can also avoid heavy server load due to file uploading and possible attacks in uploaded files.

Another improvement over user experience is auto expanding of TinyMCE editing area as feedback from students mentioned that their \textit{Read me} and \textit{Project Log} are usually quite lengthy. So auto expanding would not require too much scrolling during creating/editing a submission.
