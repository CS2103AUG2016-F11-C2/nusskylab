\chapter{Introduction}

Orbital is the School of Computing’s self-driven programming summer experience. It is designed to give first-year students the opportunity to 1) self-learn and 2) build something useful. It is designed as a 4 modular credit (MC) module that is taken pass/fail (CS/CU) over the summer1.

For the evaluation of Orbital program, students are supposed to submit to Milestones as a team, stating what they have done during that phase. And then assigned peer teams(each team will be assigned about 3 peer teams) will be giving feedback regarding the submission and the application built by the team of students. At the same time, there will be an adviser who will overlook the whole process and provide evaluation of a team's submission too. After 3 submissions and evaluations are done, a feedback is expected from a team to its peer teams and it adviser regarding the quality of evaluations received.

The nature of Orbital defines the scope of Skylab — A development project built for peer evaluations. It also provides students with a real-life Software Engineering training ground to learn and sharpen coding and system designing skills. 

\section{Challenges}

System design:
Evolving requirements:
Data migration:
Developing process:
Coding quality/maintainability:
System administration: 

\subsection{A Subsection}

Just leave it here

\section{Importance of Work}

To be continued

\section{Objectives}

To be continued