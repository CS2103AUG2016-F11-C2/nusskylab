\chapter{Testing} \label{testing}

Testing is very important in development as it serves as the validation and verification of the program\cite{citationtesting}. And also in the process of continuous development, having tests can help us catching errors, especially regression errors. Skylab as a software engineering project, is covered by different level of tests to make sure the delivered features are as promised.

Testing in GitHub flow —-- the development process adopted in Skylab —-- has a special place as well. Through continuous integration tools like Travis, each pull requests and commits on master branch will be tested and the status will be used as the indicator whether merging to master brunch should be done and the health of current master brunch.

Rspec is the testing library used in Skylab and it is widely used the testing tool for many Ruby on Rails programmers as well\cite{citationrspec}. And for database testing, FactoryGirl is used for creating records in databases. The syntax of FactoryGirl and Rspec is pretty expressive and concise, which aligns well with design of Ruby on Rails and helps to improve readability of testing code.

\section{Unit testing} \label{unittesting}

Models are the most fundamental components in Ruby on Rails` MVC pattern and they are tested and covered fully in Skylab. In Skylab, models` testing are executed on testing database instead of faking models to test logic and validation in models only. Although this would slow down execution of test suite overall as database access is usually slow, we want to make the integration of models` logic and constraints in database just would work as expected. And since currently the test suite of Skylab does not take more than 2 minutes to run, we would favor better tested code over faster execution of tests.

Controllers usually do some complicated data querying and processing so tests on controllers can fully cover as well. However, so far, helpers and mailers are not tested in Skylab now due to time limit and the fact the helpers and mailers are relatively straight forward.

\section{Acceptance testing} \label{acceptancetesting}

Acceptance testing is to make the whole system would work as expected when users carry out various of activities. Most use cases are covered in acceptance tests for Skylab like registration form submission, team invitation, admin`s overview of different models in Skylab, students` submissions, peer evaluations and feedback. Although the testing coverage in Skylab is pretty high and most use cases are included in acceptance testing as well, there are more details to check in testing and other cases like advisers` evaluations, management of evaluatings and others as well. Trying to cover more perspectives in Skylab is therefore one of the main future tasks.
