\chapter{Security} \label{security}

During the development of Skylab, security has always a priority. Common attacks such as SQL injection, XSS attacks and CSRF attacks are all partially addressed by the Rails framework itself. Skylab's main security focus is to make sure authentication process and access control are well-defined and can withstand various attacks.

\section{User authentication}

There are basically 2 ways to login to Skylab: email and password combination and NUS OpenID. For email and password combination, we employ Devise to manage authentication related information in \textit{User} model. Devise has been used widely in Rails community and currently there are more than 13,000 stars of Devise repository on GitHub\cite{citationdevise}. What is more, according to security scanning of Devise by Hakiri there is no security warning, which further proves the trust-ability of Devise\cite{citationdevisehakiri}. As for NUS OpenID, it is using OAuth 2.0 as authentication protocol, which is believed to be safe and currently is adopted by many OpenID providers as well\cite{citationnusopenid}.

\section{Access control}

Skylab is adopting role based access control strategy to grant different permissions for different users. There are basically 4 levels of checking for each incoming requests:

\begin{itemize}
  \item \textbf{login\_required:} if a action is login\_required then a user must sign to be granted for the access.
  \item \textbf{admin\_only:} if a action is admin\_only then only users with role of \textit{Admin} can access; but if current user is admin, the next check will be skipped and return true directly as admin users can access any resources in Skylab.
  \item \textbf{allowed\_users:} if a action requires login and is not only accessed by admins, then the current logged-in user will be checked against the list of allowed users who have the permissions.
% Min: define lambda or reword
  \item \textbf{strategy:} if a lambda is provided by the caller, it will be executed and whether users in allowed\_users can actually access the resource or not depends on the returned result.
\end{itemize}

% Min: Mention hakiri security warnings and faults?  Screenshot?

As this checking procedure is a common to nearly all actions, it is defined in \textit{ApplicationController} which will be effectively inherited by all controllers in Skylab. So all methods in those controllers can invoke this method to check user`s permission first.
