\chapter{Administrative Portal} \label{adminportal}

Administrative users are super users that monitor and control Skylab to monitor progress of students and performance of different teams and advisers. Therefore they have the highest privileges in Skylab. And due to the fact that admin users have so much to do in Skylab, a lot of effort has been put into the development and improvement of functionalities related to administration.

\section{Overview and export of data} \label{adminoverview}

To oversee and manage different perspectives of Skylab, admin users need to have access to all information in Skylab in a easy way. In Skylab, the basic steps for an admin user to have an overview of something is simply to navigate through navigation header and after that a table consisting of brief information about each record in the collection will be presented in a table's form. For example, if the admin user wants to view all the students in current cohort, he can simply press the link in the navigation header and students in the current cohort will be listed with basic information such as name, email, team, adviser, is\_pending. Also for each student, the admin user can get more information by clicking view to go to the student's page or edit the student's information.

Sorting of records in the table is provided for admin users to quickly rank students by simple criteria. However, filtering is not currently implemented, as admin users are considered to be skillful and familiar with the data, and mainly use the ``Find'' functionality provided by the browser to locate records.

To enable admin users to do more work, we implemented an export function for student and team data. After getting the data exported as a comma separated values (CSV), admin users can use text editors or Excel to manipulate the data further. Although ideally some features could be added to Skylab so that admin users can just do it in the system without extra steps, due to the time limit and other tasks, exporting data is a quick way to get admin users to carry out desired tasks.  Such a facility also enables future use cases beyond what is currently envisioned with Skylab.

\section{Preview as other users} \label{adminpreviewas}

Admin users can view any user's information for review.  Additionally, there is the added functionality for admin to ``(simulate) fake login as'' any other user; such that the admin session continues with the credentials of the user being simulated as. During the development of Skylab, checking student's view and adviser's view was needed in development to more easily verify problems and proofcheck their fixes.  What is more, admin users can also use this feature to actually go through the user's workflows and to assess whether they need adjustment or not.  Through this functionality, good suggestions about the user experience enhancement have been vetted by admin users. Besides these, ``fake login as'' can help admin see the issues reported by students and advisers in production, such that the code fix can developed, tested and deployed faster.

\section{Mailing} \label{adminmailing}

Mailing is a very important feature for administrative users, as many reminders and announcements need to be delivered to users' emails for them to check.  While Orbital also relies on other external methods for reminders (i.e., Facebook, Slack and Wordpress website), in practice multiple methods need to be used.  There are then basically two types of emails sent in Skylab:

\begin{itemize}
  \item {\bf System-generated reminder emails}: these emails do not require admin users' actions to be sent to users; e.g., forget-password emails to reset password, welcome email when admin users create a user, reminder emails when deadlines are coming for submission, peer evaluation and feedback.
  \item {\bf User-initiated emails}: this type of emails are written by users and sent at users specified timings and would serve as the channel for general reminders and announcements. And these emails will be recored as history so that user can look at past records to compose a similar email. This feature is not limited to admin users and in fact advisers are also allowed to use this to remind students in his/her evaluation group.
\end{itemize}

Admin users can make use of mailing to remind students and this can save administrative time and effort in manually sending emails. It can also serve as an alternative to existing communication tools (i.e., Slack) when students ignore those discussions.

