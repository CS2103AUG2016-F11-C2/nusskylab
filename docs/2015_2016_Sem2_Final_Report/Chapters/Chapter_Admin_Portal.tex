\chapter{Admin Portal} \label{adminportal}

Admin users are supposed to overlook everything in Skylab to monitor progress of students and performance of different teams and advisers. Therefore they have the highest privileges in Skylab. And due to the fact that admin users have so much to do in Skylab, a lot of effort has been put into the development and improvement of functionalities related to administration.

\section{Overview and export of data} \label{adminoverview}

To oversee and manage different perspectives of Skylab well, admin users need to have access to all information in Skylab in a easy way. In Skylab, the basic steps for an admin user to have an overview of something is simply to navigate through navigation header and after that a table consisting of brief information about each record in the collection will be presented in a table`s form. For example, if the admin user wants to view all the students in current cohort, he can simply press the link in the navigation header and students in the current cohort will be listed with basic information such as name, email, team, adviser, is\_pending. Also for each student, the admin user can get more information by clicking view to go to the student`s page or edit the student`s some information.

Sorting of records in the table is provided for admin users to quickly rank students via some condition. However, filter is not implemented as of now as admin users are considered to be skillful and familiar with the data and mainly use ``Find'' functionality provided by the browser to locate some record.

To enable admin users to do more things with the data, exporting feature has been implemented for students and teams as well. After getting data exported as CSV, admin users can use text editors or Excel to manipulate the data further. Although ideally some features could be added to Skylab so that admin users can just do it in the system without extra steps, due to the time limit and other tasks, exporting data is a quick way to get admin users to carry out desired tasks.

\section{Preview as other users} \label{adminpreviewas}

Admin users can view any user`s information for overlooking. And there is not more functionality for admin to ``fake login as'' any other user. During development of Skylab, checking student`s view and adviser`s view is needed by the developers a lot to see whether user interface is in the desired way. What is more, admin users can also use this feature to actually go through flows of user`s action needs adjustment or not and through the implementation of Skylab, many good suggestions about user experience enhancement have been by admin users thanks to this. Besides these, ``fake login as'' can help admin see the issues reported by students and advisers in production environment so that the fixing can be faster and more targeted.  

\section{Mailing} \label{adminmailing}

Mailing is a very important feature for admin users as many reminders and announcements should be delivered to users` emails for them to check. So there are basically 2 types of emails sent in Skylab.

\begin{itemize}
  \item Reminder emails: this sort of emails do not require admin users` actions to be sent to users, for example, forget-password emails to reset password, welcome email when admin users create a user, reminder emails when deadlines are coming for submission, peer evaluation and feedback.
  \item User-initiate emails: this type of emails are written by users and sent at users specified timings and would serve as the channel for general reminders and announcements. And these emails will be recored as history so that user can look at past records to compose a similar email. This feature is not limited to admin users and in fact advisers are also allowed to use this to remind students in his/her evaluation group.
\end{itemize}

Admin users can make use of mailing to remind students and this can save admin`s time and effort in manually sending emails. It can also serve as an alternative to existing communication tools such as Slack in case some students ignored those discussions.
